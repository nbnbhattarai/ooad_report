\documentclass[a4paper,12pt,onepage]{article}
% \usepackage[left=35mm, right=20mm, top=35mm, bottom=20mm, includefoot]{geometry}
\usepackage{graphicx}
\usepackage{lipsum}
\usepackage{final_report}
\usepackage{float}

\usepackage{fancyhdr}
\pagestyle{fancy}
\fancyhead{}
\fancyfoot{}
\fancyhead[R]{\thepage\ \hspace{1pt} }

\renewcommand{\headrulewidth}{0pt}
\renewcommand{\footrulewidth}{0pt}

\AtBeginDocument{\renewcommand\contentsname{Table of Contents}}

\begin{document}

% Title Page start
\pagenumbering{roman}
\addcontentsline{toc}{section}{Title Page}
\begin{titlepage}
  \centering
  \includegraphics[width=0.2\textwidth]{tu_logo.png}\par% \vspace{1cm}
  {\textsc\LARGE TRIBHUVAN UNIVERSITY \par}
  {\textsc\LARGE IOE, CENTRAL CAMPUS PULCHOWK \par}
  {\textsc\LARGE LALITPUR \par}
  \vspace{1cm}
	{\LARGE\textsc Automatic Speech Recognition System\par}
	\vspace{3cm}
        {
          {\Large Bikash Gupta (070/BCT/512)}\\
          \vspace{0.1cm}
          {\Large Kshitiz Shrestha (070/BCT/518)}\\
          \vspace{0.1cm}
          {\Large Miran Ghimire (070/BCT/521)}\\
          \vspace{0.1cm}
          {\Large Nabin Bhattarai (070/BCT/522)}\\
          \vspace{0.1cm}
          {\Large Pabin Raj Luitel (070/BCT/523)}\\
          \vspace{0.1cm}
          {\Large Sabin Silwal (070/BCT/532)}\\
          \vspace{0.1cm}
        }
	\vspace{1cm}

	\vfill

        % Bottom of the page
	{\large \today\par}
\end{titlepage}
% Title Page End
\setcounter{page}{2}

\section*{Acknowledgement}
	We would like to express our sincere gratitude Department of Electronics and Computer Engineering of I.O.E, Pulchowk Campus for providing us opportunity to implement the knowledge of Object Oriented And Design(OOAD) on the project.
	\\
	\\
	Besides we are highly indebted to our teacher Dr. Arun Timalsina for his guidance on subject of OOAD and motivating us on project. Furthumore we would like to acknowledge the crucial role of our teacher Arun Verma whose contribution in stimulating suggestions and encouragement helped us to coordinate our project.
	\\
	\\
	Finally we would like to appreciate the guidance of our friends and all those people who helped us directly or indirectly to complete our project.
\addcontentsline{toc}{section}{Acknowledgement}

\cleardoublepage
\section*{Abstract}
\addcontentsline{toc}{section}{Abstract}
	With the aim of overcoming the traditional way of interacting with the computerized system or devices 'Automatic Speech Recognition System' has been introduced. It is a software that which feature the more convinent and more natural way to interact with the computer system via the voice.
Using this software, one can get his speech recognized easily. Thus 'Automatic Speech Recognition Sytem' is a software that automatically recognizes the voice of the user and convert it into a text. Besides this application, can have a wide range of application like voice recognition, voice controlled system etc.
\cleardoublepage

{
  \setlength{\parskip}{0em}
  \renewcommand\contentsname{Table of Contents} % This will change heading text
  \tableofcontents \addcontentsline{toc}{section}{Table of Contents}
}

\cleardoublepage
\setlength{\parskip}{0em}
\addcontentsline{toc}{section}{List Of Figures}
\listoffigures

\newpage

\section*{LIST OF ABBREVIATIONS}

\begin{tabular}{l l}
  ASR & Automatic Speech Recognition \\
  AI & Artificial Intelligence \\
\end{tabular}

\cleardoublepage
\pagenumbering{arabic}
\section{OverView}

\subsection{Introduction}
 With the pace of the technological developement, people these days expect to do every of their computer work in a more simpler, convinent and natural way. By realizing the need of the general people we have created a prototype 
'Automatic Speech Recognition System', that can convert a speech of the user into the corresponding text.
\\
\\
'Automatic Speech Recognition System' is prototype that one can use to convert his/her speech into a corresponding text. Initially a voice of user is recorded or user are asked to say something. Voice is recorded with the sampling rate of 16000Hz for certain timeperiod. Thus obtained record 
usually comes with the noises, therefore it has to be filtered. Filter is done by the software itself. After wards
\\
\\
Thus the prototype developed basically converts the user-speech into a text. However it can be equally be suitable in other voiced system such as 
speaker recognition, voice controlled system etc.
\newpage
\subsection{Statement of the problem}

\newpage
\subsection{Objectives}
The primary objective of the project was to develop and implement a fully functional ASR system using the concept of OOAD.The whole process of creating the system was achieved through a series of minor procedures which can be furthur subdivided into two major stages. The first involved the problem study,solution propostion, designing and development of system. The second phase included actual operation, solution of obsolescence problems and progress towards the new standards all of which are enlisted below:
\\
\begin{enumerate}
	\item \textbf{Stage 1}
	\begin{itemize}
	\item To design and develop the system according to the concept of OOAD.
	\item To implement a system once it was developed and put into the service.
	\end{itemize}
\item \textbf{Stage 2}
	\begin{itemize}
	\item Consolidating the system by removing the trivial features and small as well as big bugs.
	\item Removing the obsolescence problems.
	\item To revise the sytem according to need and change of standard in use.
	\end{itemize}
\end{enumerate}
\newpage
\section{Literature Review} 
	\subsection{Proposed System}
	The system would be able to replace the conventional method of interacting with the computerized system. The proposed system basically has the following the features.\\
	\begin{itemize}
	\item Recording of speech.
	\item Conversion of speech to text.
	\end{itemize}
	\newpage
	
	\subsection{Project Description}
	\emph{'Automatic Speech Recognition System'} is a prototype developed with the aim of making the interaction with the computerized system as natural as possible by giving the voice commands. However the system developed only comprises of converting a speech to text which is a very first step toward the goal. Currently our system is capable of converting the speech into the text.\\
Our system comprises of the two yet interdependent subsystems.
\begin{itemize}
\item \textbf{Front End} 
\par
It is the visible part of the system. The user will interact with this part of the system. It offers the system interface and consist of the request to give a speech.
\item \textbf{Back End}
\par
This is the part of the system where every processing work is done. This is the part where we train the neural network for recoginizing the words.This unit is handled by the developer who is responsible for the addition, modification and updating the training content of the neural network. Currently forwords which are unable to process by our neural network, google API has beenused for maintaining the system accuracy.
\end{itemize}
	\newpage
\section{Methodology}
The prototype was developed using language python.
\subsection{Design Approach}
As software developement lifecycle of our system is linear and sequential along with the number of people involved in project 'Rapid Application Developemnt(RAD)' was the model followed for a software development.
\subsubsection{Developement Process: Macro level}
\begin{enumerate}
\item Conceptualization
\newpage
\item Analysis
\newpage
\item Design
\newpage
\item Evolution
\newpage
\item Maintainance
\newpage
\end{enumerate}
\subsubsection{Developement Process: Micro level}
\begin{enumerate}
\item Identification of Class and Object
\newpage
\item Identification of semantics of Class and Object
\newpage
\item Identification of realationship of Class and Object
\newpage
\item Implementing Classes and Objects
\newpage
\end{enumerate}
\subsection{UML Diagram}
The UML diagram of the system is as follows.

\newpage
\subsection{Dataflow Diagram}
The Dataflow diagram of the system is as follows.

\newpage
\subsection{Class Diagram}\par
The Class diagram of the system is as follows.

\newpage
\subsection{Activity Diagram}
\newpage
\subsection{Sequence Diagram}
\newpage

\section{Result}
\newpage
\section{Discussion}

\newpage
\section{Further Improvement}

\newpage
\section{Conclusion}

\newpage
\bibliographystyle{ieeetr}
\bibliography{final_report}
\nocite{*}
\end{document}

%%% Local Variables:
%%% mode: latex
%%% TeX-master: "final_report"
%%% End:
